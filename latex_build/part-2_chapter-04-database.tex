\chapter{Database Design with SQLite}\label{database-design-with-sqlite}

\begin{center}\rule{0.5\linewidth}{0.5pt}\end{center}

\section{4.1 Why SQLite for
Production?}\label{why-sqlite-for-production}

SQLite is often dismissed as ``just for prototyping.'' This is a myth.

\subsection{SQLite in Production}\label{sqlite-in-production}

These companies use SQLite in production:

\begin{longtable}[]{@{}ll@{}}
\toprule\noalign{}
Company & Use Case \\
\midrule\noalign{}
\endhead
\bottomrule\noalign{}
\endlastfoot
\textbf{Apple} & Every iPhone, iPad, Mac \\
\textbf{Airbus} & Flight software \\
\textbf{Dropbox} & Desktop client \\
\textbf{WhatsApp} & Message storage \\
\textbf{Adobe} & Creative Cloud \\
\end{longtable}

\subsection{The Numbers}\label{the-numbers}

SQLite handles: - \textbf{100,000+} simultaneous readers -
\textbf{10,000+} writes per second - \textbf{281 TB} maximum database
size - \textbf{ACID} compliant transactions

\subsection{When NOT to Use SQLite}\label{when-not-to-use-sqlite}

\begin{itemize}
\tightlist
\item
  High-write workloads with many concurrent writers
\item
  Need for replication across multiple servers
\item
  Very large datasets (\textgreater{} 1TB)
\item
  Real-time synchronization requirements
\end{itemize}

\subsection{When SQLite Shines}\label{when-sqlite-shines}

\begin{itemize}
\tightlist
\item
  Read-heavy workloads (like content delivery)
\item
  Single-server deployments
\item
  Embedded applications
\item
  Simplicity is a priority
\end{itemize}

Our educational platform is \textbf{read-heavy} (many users reading
content, few admins writing). SQLite is perfect.

\begin{center}\rule{0.5\linewidth}{0.5pt}\end{center}

\section{4.2 Single File vs Multiple
Databases}\label{single-file-vs-multiple-databases}

\subsection{The Traditional Approach}\label{the-traditional-approach}

Most applications use one database for everything:

\begin{figure}
\centering
\includegraphics[width=0.6\linewidth,height=\textheight,keepaspectratio]{../static/images/chapter-04/single-db.png}
\caption{Traditional Single Database}
\end{figure}

\subsection{Our Approach: Multiple
Databases}\label{our-approach-multiple-databases}

\begin{figure}
\centering
\includegraphics[width=0.6\linewidth,height=\textheight,keepaspectratio]{../static/images/chapter-04/dual-db.png}
\caption{Multiple Databases Strategy}
\end{figure}

\subsection{Why Separate?}\label{why-separate}

\begin{longtable}[]{@{}ll@{}}
\toprule\noalign{}
Benefit & Explanation \\
\midrule\noalign{}
\endhead
\bottomrule\noalign{}
\endlastfoot
\textbf{Independent Backups} & Backup user data hourly, content
weekly \\
\textbf{Easy Updates} & Replace content.db without touching users \\
\textbf{Clear Ownership} & Content team manages content.db \\
\textbf{Performance} & Content reads don't lock user writes \\
\textbf{Security} & Content can be read-only in production \\
\end{longtable}

\begin{center}\rule{0.5\linewidth}{0.5pt}\end{center}

\section{4.3 Schema Design Principles}\label{schema-design-principles}

Good schema design prevents pain later.

\subsection{Principle 1: Use TEXT for IDs (in
content.db)}\label{principle-1-use-text-for-ids-in-content.db}

\begin{lstlisting}[language=SQL]
-- Instead of this:
CREATE TABLE lessons (
    id INTEGER PRIMARY KEY  -- What is lesson 42?
);

-- Do this:
CREATE TABLE lessons (
    id TEXT PRIMARY KEY     -- lesson "hadith-jibril" is clear
);
\end{lstlisting}

\textbf{Why?} Human-readable IDs are: - Self-documenting - Easier to
debug - URL-friendly (slugs) - Content-addressable

\subsection{Principle 2: Use INTEGER for Auto-Increment (in
users.db)}\label{principle-2-use-integer-for-auto-increment-in-users.db}

\begin{lstlisting}[language=SQL]
-- For user-generated data, auto-increment is fine:
CREATE TABLE users (
    id INTEGER PRIMARY KEY AUTOINCREMENT
);
\end{lstlisting}

\textbf{Why?} User IDs don't need to be memorable.

\subsection{Principle 3: Store Timestamps as
TEXT}\label{principle-3-store-timestamps-as-text}

\begin{lstlisting}[language=SQL]
-- SQLite has no native date type
created_at TEXT DEFAULT CURRENT_TIMESTAMP
-- Stores as: "2024-12-11 20:00:00"
\end{lstlisting}

\textbf{Why?} ISO 8601 strings are: - Human-readable - Sortable as text
- Timezone-aware compatible

\subsection{Principle 4: Use INTEGER for
Booleans}\label{principle-4-use-integer-for-booleans}

\begin{lstlisting}[language=SQL]
-- SQLite has no native boolean
is_published INTEGER DEFAULT 0  -- 0 = false, 1 = true
\end{lstlisting}

\subsection{Principle 5: Store JSON as
TEXT}\label{principle-5-store-json-as-text}

\begin{lstlisting}[language=SQL]
-- For complex data like quizzes
quiz TEXT  -- JSON array of questions
\end{lstlisting}

\textbf{Parse in application code:}

\begin{lstlisting}
const questions = JSON.parse(video.quiz || '[]');
\end{lstlisting}

\begin{center}\rule{0.5\linewidth}{0.5pt}\end{center}

\section{4.4 Hierarchical Data
Modeling}\label{hierarchical-data-modeling}

Our content has a natural hierarchy:

\begin{figure}
\centering
\includegraphics[width=0.6\linewidth,height=\textheight,keepaspectratio]{../static/images/chapter-04/hierarchical-data.png}
\caption{Content Hierarchy}
\end{figure}

\subsection{The Schema}\label{the-schema}

\begin{lstlisting}[language=SQL]
-- Level 1: Tracks (Learning Paths)
CREATE TABLE tracks (
    id TEXT PRIMARY KEY,          -- "hadith"
    title TEXT NOT NULL,          -- "Hadith Studies"
    description TEXT,
    icon TEXT,                    -- "📜"
    "order" INTEGER DEFAULT 0
);

-- Level 2: Series (Course Collections)
CREATE TABLE series (
    id TEXT PRIMARY KEY,          -- "nawawi-40"
    track_id TEXT REFERENCES tracks(id),
    title TEXT NOT NULL,          -- "40 Nawawi Hadith"
    description TEXT,
    instructor TEXT,              -- "Sheikh X"
    thumbnail_url TEXT,
    "order" INTEGER DEFAULT 0
);

-- Level 3: Lessons (Individual Classes)
CREATE TABLE lessons (
    id TEXT PRIMARY KEY,          -- "hadith-jibril"
    track_id TEXT REFERENCES tracks(id),
    series_id TEXT REFERENCES series(id),
    title TEXT NOT NULL,
    slug TEXT UNIQUE NOT NULL,    -- URL-friendly
    description TEXT,
    instructor TEXT,
    pdf_url TEXT,
    is_published INTEGER DEFAULT 0,
    "order" INTEGER DEFAULT 0
);

-- Level 4: Videos (Lesson Segments)
CREATE TABLE videos (
    id TEXT PRIMARY KEY,          -- "hadith-jibril-v1"
    lesson_id TEXT REFERENCES lessons(id),
    title TEXT,
    youtube_id TEXT,              -- YouTube video ID
    download_url TEXT,            -- Direct download link
    duration INTEGER,             -- Seconds
    summary TEXT,
    quiz TEXT,                    -- JSON array
    "order" INTEGER DEFAULT 0
);
\end{lstlisting}

\subsection{Querying the Hierarchy}\label{querying-the-hierarchy}

\textbf{Get full lesson with related data:}

\begin{lstlisting}
// Get lesson with track info
const lesson = db
    .select()
    .from(lessons)
    .leftJoin(tracks, eq(lessons.trackId, tracks.id))
    .where(eq(lessons.slug, 'hadith-jibril'))
    .get();

// Get videos for lesson
const videos = db
    .select()
    .from(videos)
    .where(eq(videos.lessonId, lesson.id))
    .orderBy(videos.order)
    .all();
\end{lstlisting}

\begin{center}\rule{0.5\linewidth}{0.5pt}\end{center}

\section{4.5 Audit Trail
Implementation}\label{audit-trail-implementation}

Track who changed what and when.

\subsection{The Fields}\label{the-fields}

Add these to every content table:

\begin{lstlisting}[language=SQL]
ALTER TABLE lessons ADD COLUMN created_at TEXT DEFAULT CURRENT_TIMESTAMP;
ALTER TABLE lessons ADD COLUMN updated_at TEXT;
ALTER TABLE lessons ADD COLUMN last_modified_by TEXT;  -- "script" or user ID
ALTER TABLE lessons ADD COLUMN is_locked INTEGER DEFAULT 0;
\end{lstlisting}

\subsection{Complete Table Definition}\label{complete-table-definition}

\begin{lstlisting}[language=SQL]
CREATE TABLE lessons (
    -- Core fields
    id TEXT PRIMARY KEY,
    track_id TEXT,
    series_id TEXT,
    title TEXT NOT NULL,
    slug TEXT UNIQUE NOT NULL,
    description TEXT,
    instructor TEXT,
    pdf_url TEXT,
    is_published INTEGER DEFAULT 0,
    "order" INTEGER DEFAULT 0,
    
    -- Audit fields
    created_at TEXT DEFAULT CURRENT_TIMESTAMP,
    updated_at TEXT,
    last_modified_by TEXT DEFAULT 'script',
    is_locked INTEGER DEFAULT 0,
    
    -- Foreign keys
    FOREIGN KEY (track_id) REFERENCES tracks(id),
    FOREIGN KEY (series_id) REFERENCES series(id)
);
\end{lstlisting}

\subsection{Using the Audit Trail}\label{using-the-audit-trail}

\textbf{When updating via script:}

\begin{lstlisting}
db.update(lessons)
    .set({
        title: newTitle,
        updated_at: new Date().toISOString(),
        last_modified_by: 'script'
    })
    .where(eq(lessons.id, lessonId))
    .run();
\end{lstlisting}

\textbf{When updating via admin:}

\begin{lstlisting}
db.update(lessons)
    .set({
        title: newTitle,
        updated_at: new Date().toISOString(),
        last_modified_by: userId.toString()  // Human user
    })
    .where(eq(lessons.id, lessonId))
    .run();
\end{lstlisting}

\textbf{Protecting human edits:}

\begin{lstlisting}
function shouldUpdate(existing) {
    // Don't overwrite if locked
    if (existing.isLocked) return false;
    
    // Don't overwrite human edits (unless forced)
    if (existing.lastModifiedBy !== 'script') {
        return FORCE_MODE;
    }
    
    return true;
}
\end{lstlisting}

\begin{center}\rule{0.5\linewidth}{0.5pt}\end{center}

\section{4.6 Indexing Strategies}\label{indexing-strategies}

Indexes speed up queries but slow down writes. Use wisely.

\subsection{When to Index}\label{when-to-index}

\begin{longtable}[]{@{}ll@{}}
\toprule\noalign{}
Scenario & Index? \\
\midrule\noalign{}
\endhead
\bottomrule\noalign{}
\endlastfoot
Column in WHERE clause & Yes \\
Column in JOIN & Yes \\
Column in ORDER BY & Maybe \\
Rarely queried column & No \\
Frequently updated column & Be careful \\
\end{longtable}

\subsection{Our Indexes}\label{our-indexes}

\begin{lstlisting}[language=SQL]
-- users.db
CREATE INDEX idx_sessions_user ON sessions(user_id);
CREATE INDEX idx_progress_user ON lesson_progress(user_id);
CREATE INDEX idx_progress_lesson ON lesson_progress(lesson_id);

-- content.db
CREATE INDEX idx_lessons_track ON lessons(track_id);
CREATE INDEX idx_lessons_series ON lessons(series_id);
CREATE INDEX idx_lessons_slug ON lessons(slug);
CREATE INDEX idx_videos_lesson ON videos(lesson_id);
CREATE INDEX idx_series_track ON series(track_id);
\end{lstlisting}

\subsection{Composite Indexes}\label{composite-indexes}

For queries with multiple conditions:

\begin{lstlisting}[language=SQL]
-- For: WHERE user_id = ? AND lesson_id = ?
CREATE UNIQUE INDEX idx_progress_user_lesson 
    ON lesson_progress(user_id, lesson_id);
\end{lstlisting}

\begin{center}\rule{0.5\linewidth}{0.5pt}\end{center}

\section{4.7 WAL Mode and Performance}\label{wal-mode-and-performance}

WAL (Write-Ahead Logging) improves SQLite concurrency.

\subsection{Enabling WAL Mode}\label{enabling-wal-mode}

\begin{lstlisting}
const db = new Database('data/users.db');
db.exec('PRAGMA journal_mode = WAL');
\end{lstlisting}

\subsection{What WAL Does}\label{what-wal-does}

\textbf{Without WAL:} - Readers block writers - Writers block readers -
Only one writer at a time

\textbf{With WAL:} - Readers never block writers - Writers never block
readers - Multiple concurrent readers

\subsection{Other Performance PRAGMAs}\label{other-performance-pragmas}

\begin{lstlisting}
// Apply these at connection time
db.exec('PRAGMA journal_mode = WAL');       // Enable WAL
db.exec('PRAGMA busy_timeout = 5000');      // Wait 5s if busy
db.exec('PRAGMA synchronous = NORMAL');     // Balance safety/speed
db.exec('PRAGMA foreign_keys = ON');        // Enforce FK constraints
db.exec('PRAGMA cache_size = -64000');      // 64MB cache
\end{lstlisting}

\subsection{When to Use WAL}\label{when-to-use-wal}

\begin{longtable}[]{@{}ll@{}}
\toprule\noalign{}
Scenario & Mode \\
\midrule\noalign{}
\endhead
\bottomrule\noalign{}
\endlastfoot
Read-heavy (content.db) & WAL \\
Write-heavy (users.db) & WAL \\
Single reader/writer & DELETE (default) \\
Network storage & DELETE (WAL needs local) \\
\end{longtable}

\begin{center}\rule{0.5\linewidth}{0.5pt}\end{center}

\section{Summary}\label{summary}

In this chapter, we designed our database architecture:

\begin{longtable}[]{@{}ll@{}}
\toprule\noalign{}
Aspect & Decision \\
\midrule\noalign{}
\endhead
\bottomrule\noalign{}
\endlastfoot
Engine & SQLite (production-ready) \\
Strategy & Dual database (users + content) \\
IDs & TEXT for content, INTEGER for users \\
Hierarchy & Tracks → Series → Lessons → Videos \\
Auditing & created\_at, updated\_at, last\_modified\_by, is\_locked \\
Performance & WAL mode, strategic indexes \\
\end{longtable}

Our schema is now ready. In the next chapter, we'll set up Drizzle ORM
to interact with these databases in a type-safe way.

\begin{center}\rule{0.5\linewidth}{0.5pt}\end{center}

\begin{quote}
\textbf{Next Chapter}: \href{./chapter-05-drizzle.md}{Chapter 5: Drizzle
ORM Setup}
\end{quote}
