\chapter{Setting Up the Development
Environment}\label{setting-up-the-development-environment}

\begin{center}\rule{0.5\linewidth}{0.5pt}\end{center}

\section{3.1 Installing Bun}\label{installing-bun}

Bun is our JavaScript runtime --- think of it as a faster, more modern
alternative to Node.js.

\subsection{What is Bun?}\label{what-is-bun}

Bun is an all-in-one toolkit that includes: - \textbf{Runtime}: Executes
JavaScript/TypeScript - \textbf{Package Manager}: Faster than npm/yarn -
\textbf{Bundler}: Built-in bundling for production - \textbf{Test
Runner}: Native testing support

\subsection{Installation}\label{installation}

\textbf{macOS / Linux:}

\begin{lstlisting}[language=bash]
curl -fsSL https://bun.sh/install | bash
\end{lstlisting}

\textbf{Windows (via WSL):}

\begin{lstlisting}[language=bash]
# First, install WSL if you haven't
wsl --install

# Then install Bun in WSL
curl -fsSL https://bun.sh/install | bash
\end{lstlisting}

\textbf{Verify installation:}

\begin{lstlisting}[language=bash]
bun --version
# Should output: 1.x.x
\end{lstlisting}

\subsection{Why Bun Over Node.js?}\label{why-bun-over-node.js}

\begin{longtable}[]{@{}lll@{}}
\toprule\noalign{}
Feature & Node.js & Bun \\
\midrule\noalign{}
\endhead
\bottomrule\noalign{}
\endlastfoot
Startup time & \textasciitilde40ms & \textasciitilde5ms \\
Package install & \textasciitilde10s & \textasciitilde2s \\
TypeScript & Needs transpiler & Native \\
SQLite & Needs npm package & Built-in \\
Test runner & Needs Jest/Vitest & Built-in \\
\end{longtable}

\begin{center}\rule{0.5\linewidth}{0.5pt}\end{center}

\section{3.2 Editor Setup (VS Code /
Cursor)}\label{editor-setup-vs-code-cursor}

A good editor setup will save you hours of debugging.

\subsection{VS Code Installation}\label{vs-code-installation}

Download from: https://code.visualstudio.com/

\subsection{Cursor (AI-Powered
Alternative)}\label{cursor-ai-powered-alternative}

Cursor is VS Code with AI built-in: https://cursor.sh/

\subsection{Recommended Settings}\label{recommended-settings}

Create \passthrough{\lstinline!.vscode/settings.json!} in your project:

\begin{lstlisting}
{
    // Editor basics
    "editor.tabSize": 4,
    "editor.formatOnSave": true,
    "editor.defaultFormatter": "esbenp.prettier-vscode",
    
    // TypeScript
    "typescript.preferences.importModuleSpecifier": "relative",
    "typescript.suggest.autoImports": true,
    
    // Svelte
    "svelte.enable-ts-plugin": true,
    
    // File associations
    "files.associations": {
        "*.svelte": "svelte"
    },
    
    // Exclude from search
    "search.exclude": {
        "**/node_modules": true,
        "**/dist": true,
        "**/.svelte-kit": true
    }
}
\end{lstlisting}

\begin{center}\rule{0.5\linewidth}{0.5pt}\end{center}

\section{3.3 Essential Extensions}\label{essential-extensions}

Install these VS Code extensions:

\subsection{Must-Have}\label{must-have}

\begin{longtable}[]{@{}ll@{}}
\toprule\noalign{}
Extension & Purpose \\
\midrule\noalign{}
\endhead
\bottomrule\noalign{}
\endlastfoot
\textbf{Svelte for VS Code} & Svelte language support \\
\textbf{Tailwind CSS IntelliSense} & Tailwind autocomplete \\
\textbf{Prettier} & Code formatting \\
\textbf{ESLint} & Code linting \\
\textbf{Error Lens} & Inline error display \\
\end{longtable}

\subsection{Recommended}\label{recommended}

\begin{longtable}[]{@{}ll@{}}
\toprule\noalign{}
Extension & Purpose \\
\midrule\noalign{}
\endhead
\bottomrule\noalign{}
\endlastfoot
\textbf{GitLens} & Git blame and history \\
\textbf{SQLite Viewer} & View .db files \\
\textbf{Thunder Client} & API testing \\
\textbf{Todo Tree} & Track TODOs \\
\end{longtable}

\subsection{Installation via CLI}\label{installation-via-cli}

\begin{lstlisting}[language=bash]
# Install all at once
code --install-extension svelte.svelte-vscode
code --install-extension bradlc.vscode-tailwindcss
code --install-extension esbenp.prettier-vscode
code --install-extension dbaeumer.vscode-eslint
code --install-extension usernamehw.errorlens
code --install-extension eamodio.gitlens
code --install-extension qwtel.sqlite-viewer
\end{lstlisting}

\begin{center}\rule{0.5\linewidth}{0.5pt}\end{center}

\section{3.4 Git Configuration}\label{git-configuration}

Version control is essential. Let's set it up properly.

\subsection{Initial Configuration}\label{initial-configuration}

\begin{lstlisting}[language=bash]
# Set your identity
git config --global user.name "Your Name"
git config --global user.email "your@email.com"

# Better defaults
git config --global init.defaultBranch main
git config --global pull.rebase true
git config --global core.autocrlf input  # Linux/Mac
\end{lstlisting}

\subsection{SSH Key Setup (for GitHub)}\label{ssh-key-setup-for-github}

\begin{lstlisting}[language=bash]
# Generate SSH key
ssh-keygen -t ed25519 -C "your@email.com"

# Start SSH agent
eval "$(ssh-agent -s)"
ssh-add ~/.ssh/id_ed25519

# Copy public key
cat ~/.ssh/id_ed25519.pub
# Paste this in GitHub → Settings → SSH Keys
\end{lstlisting}

\subsection{Project .gitignore}\label{project-.gitignore}

Create \passthrough{\lstinline!.gitignore!}:

\begin{lstlisting}
# Dependencies
node_modules/
.pnpm-store/

# Build outputs
.svelte-kit/
build/
dist/

# Environment
.env
.env.local
.env.*.local

# Databases (optional: you might want to track content.db)
data/*.db
data/*.db-wal
data/*.db-shm

# Editor
.vscode/*
!.vscode/settings.json
!.vscode/extensions.json
.idea/
*.swp
*.swo

# OS
.DS_Store
Thumbs.db

# Logs
*.log
npm-debug.log*
\end{lstlisting}

\begin{center}\rule{0.5\linewidth}{0.5pt}\end{center}

\section{3.5 Project Initialization}\label{project-initialization}

Let's create our project from scratch.

\subsection{Create Project Directory}\label{create-project-directory}

\begin{lstlisting}[language=bash]
mkdir afidna
cd afidna
\end{lstlisting}

\subsection{Initialize with Bun +
SvelteKit}\label{initialize-with-bun-sveltekit}

\begin{lstlisting}[language=bash]
# Create SvelteKit project
bun create svelte@latest .

# When prompted, choose:
# - Skeleton project
# - TypeScript
# - ESLint: Yes
# - Prettier: Yes
# - Playwright: No (we'll add later)
# - Vitest: No (we'll use Bun's test runner)
\end{lstlisting}

\subsection{Install Dependencies}\label{install-dependencies}

\begin{lstlisting}[language=bash]
# Core dependencies
bun add drizzle-orm better-sqlite3 bcrypt

# Development dependencies
bun add -d drizzle-kit @types/better-sqlite3 @types/bcrypt

# UI dependencies
bun add -d tailwindcss @tailwindcss/vite daisyui
\end{lstlisting}

\subsection{SvelteKit Configuration}\label{sveltekit-configuration}

Update \passthrough{\lstinline!svelte.config.js!}:

\begin{lstlisting}
import adapter from '@sveltejs/adapter-node';
import { vitePreprocess } from '@sveltejs/vite-plugin-svelte';

/** @type {import('@sveltejs/kit').Config} */
const config = {
    preprocess: vitePreprocess(),
    kit: {
        adapter: adapter(),
        alias: {
            $components: 'src/lib/components',
            $server: 'src/lib/server'
        }
    }
};

export default config;
\end{lstlisting}

\subsection{Tailwind Setup}\label{tailwind-setup}

Create \passthrough{\lstinline!src/app.css!}:

\begin{lstlisting}
@import 'tailwindcss';
@plugin 'daisyui';

/* RTL Support */
[dir="rtl"] {
    text-align: right;
}

/* Custom utilities */
@layer utilities {
    .card-hover {
        @apply transition-transform hover:scale-[1.02];
    }
}
\end{lstlisting}

Update \passthrough{\lstinline!vite.config.ts!}:

\begin{lstlisting}
import { sveltekit } from '@sveltejs/kit/vite';
import tailwindcss from '@tailwindcss/vite';
import { defineConfig } from 'vite';

export default defineConfig({
    plugins: [
        tailwindcss(),
        sveltekit()
    ]
});
\end{lstlisting}

\begin{center}\rule{0.5\linewidth}{0.5pt}\end{center}

\section{3.6 Directory Structure Best
Practices}\label{directory-structure-best-practices}

Let's organize our project for maintainability.

\subsection{The Complete Structure}\label{the-complete-structure}

\begin{lstlisting}
afidna/
├── src/
│   ├── lib/                      # Shared code
│   │   ├── components/           # Svelte components
│   │   │   ├── Navbar.svelte
│   │   │   ├── Footer.svelte
│   │   │   ├── VideoPlayer.svelte
│   │   │   └── Quiz.svelte
│   │   ├── server/               # Server-only code
│   │   │   ├── auth.ts           # Authentication logic
│   │   │   └── db/
│   │   │       ├── index.ts      # Database connections
│   │   │       ├── schema.ts     # users.db schema
│   │   │       └── schema-content.ts
│   │   ├── stores/               # Svelte stores
│   │   │   └── user.ts
│   │   ├── utils/                # Utility functions
│   │   │   └── format.ts
│   │   └── theme.ts              # Theme configuration
│   │
│   ├── routes/                   # SvelteKit routes
│   │   ├── +layout.svelte        # Root layout
│   │   ├── +layout.server.ts     # Layout data
│   │   ├── +page.svelte          # Homepage
│   │   ├── +page.server.ts
│   │   ├── tracks/               # /tracks routes
│   │   ├── lessons/              # /lessons routes
│   │   ├── search/               # /search route
│   │   ├── auth/                 # /auth routes
│   │   │   ├── login/
│   │   │   ├── register/
│   │   │   └── logout/
│   │   ├── admin/                # /admin routes
│   │   │   ├── +page.svelte
│   │   │   ├── lessons/
│   │   │   ├── videos/
│   │   │   └── series/
│   │   └── api/                  # API routes
│   │       └── progress/
│   │
│   ├── hooks.server.ts           # Server hooks (middleware)
│   ├── app.html                  # HTML template
│   ├── app.css                   # Global styles
│   └── app.d.ts                  # Type declarations
│
├── scripts/                      # Management scripts
│   ├── setup-db.ts               # Create users.db
│   ├── setup-content.ts          # Create content.db
│   └── seed-content.ts           # Import content
│
├── data/                         # Database files
│   ├── users.db
│   └── content.db
│
├── raw_content/                  # Content source files
│   └── content.json
│
├── static/                       # Static assets
│   ├── favicon.png
│   └── images/
│
├── book/                         # This book!
│
├── .gitignore
├── package.json
├── svelte.config.js
├── tsconfig.json
├── vite.config.ts
└── README.md
\end{lstlisting}

\subsection{Directory Purposes}\label{directory-purposes}

\begin{longtable}[]{@{}lll@{}}
\toprule\noalign{}
Directory & Purpose & Access \\
\midrule\noalign{}
\endhead
\bottomrule\noalign{}
\endlastfoot
\passthrough{\lstinline!src/lib/components/!} & Reusable UI components &
Client + Server \\
\passthrough{\lstinline!src/lib/server/!} & Server-only code & Server
only \\
\passthrough{\lstinline!src/lib/stores/!} & Reactive state & Client
only \\
\passthrough{\lstinline!src/routes/!} & Pages and API & Both \\
\passthrough{\lstinline!scripts/!} & CLI utilities & Server only \\
\passthrough{\lstinline!data/!} & Database files & Server only \\
\passthrough{\lstinline!static/!} & Static files & Public \\
\end{longtable}

\subsection{Creating the Structure}\label{creating-the-structure}

\begin{lstlisting}[language=bash]
# Create all directories
mkdir -p src/lib/{components,server/db,stores,utils}
mkdir -p src/routes/{tracks,lessons,search,auth/{login,register,logout}}
mkdir -p src/routes/admin/{lessons,videos,series}
mkdir -p src/routes/api/progress
mkdir -p scripts
mkdir -p data
mkdir -p raw_content
mkdir -p static/images
\end{lstlisting}

\begin{center}\rule{0.5\linewidth}{0.5pt}\end{center}

\section{Verification Checklist}\label{verification-checklist}

Before moving on, verify your setup:

\begin{lstlisting}[language=bash]
# Check Bun
bun --version    # Should be 1.x.x

# Check project runs
bun run dev      # Should start on localhost:5173

# Check TypeScript
bun run check    # Should pass

# Check Git
git status       # Should show your project
\end{lstlisting}

\subsection{Expected Output}\label{expected-output}

When you run \passthrough{\lstinline!bun run dev!}, you should see:

\begin{lstlisting}
  VITE v5.x.x  ready in xxx ms

  ➜  Local:   http://localhost:5173/
  ➜  Network: use --host to expose
  ➜  press h + enter to show help
\end{lstlisting}

\begin{center}\rule{0.5\linewidth}{0.5pt}\end{center}

\section{Summary}\label{summary}

In this chapter, we set up:

\begin{itemize}
\tightlist
\item
  ✅ Bun runtime
\item
  ✅ VS Code with essential extensions
\item
  ✅ Git with proper configuration
\item
  ✅ SvelteKit project
\item
  ✅ Tailwind CSS + DaisyUI
\item
  ✅ Organized directory structure
\end{itemize}

Our development environment is now ready. In the next part, we'll start
building the backend --- beginning with database design.

\begin{center}\rule{0.5\linewidth}{0.5pt}\end{center}

\begin{quote}
\textbf{Next Chapter}: \href{../part-2/chapter-04-database.md}{Part II →
Chapter 4: Database Design with SQLite}
\end{quote}

\begin{center}\rule{0.5\linewidth}{0.5pt}\end{center}

\section{Quick Reference}\label{quick-reference}

\subsection{Common Commands}\label{common-commands}

\begin{lstlisting}[language=bash]
# Start development server
bun run dev

# Type checking
bun run check

# Format code
bun run format

# Build for production
bun run build

# Preview production build
bun run preview
\end{lstlisting}

\subsection{Troubleshooting}\label{troubleshooting}

\textbf{Problem}: \passthrough{\lstinline!command not found: bun!}
\textbf{Solution}: Restart your terminal or run
\passthrough{\lstinline!source \~/.bashrc!}

\textbf{Problem}: \passthrough{\lstinline!Port 5173 already in use!}
\textbf{Solution}: Kill the process or use
\passthrough{\lstinline!bun run dev -- --port 3000!}

\textbf{Problem}: TypeScript errors in editor \textbf{Solution}: Restart
TypeScript server (Cmd/Ctrl + Shift + P → ``TypeScript: Restart TS
Server'')
